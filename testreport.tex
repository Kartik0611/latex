\documentclass[12pt,a4paper]{  report}
\usepackage[hmargin=3cm,vmargin=3cm]{geometry}
\usepackage{graphicx}
\usepackage{caption}
\usepackage{array}
\usepackage{listings}
\usepackage{hyperref}
\usepackage{mathtools}
\usepackage{textcomp}
\graphicspath{Images}
\begin{document}
\begin{figure}
\centering
\includegraphics[width = 0.3\textwidth]{iit.png}
\hspace{1cm}
\includegraphics[width = 0.4\textwidth]{fossee-logo.png}
\end{figure}

\title{\textbf{\textbf{Summer Fellowship Report}}\vspace{5mm} \\\small On \\\vspace{5mm} \textbf{\large Spice Simulations using Kicad nightly builds}\vspace{5mm} \\ \vspace{5mm}\small Submitted by\\  \vspace{5mm}  \large \textbf{Akshay NH  and Athul MS}\\ \vspace{5mm}
\small Under the guidance of \\ \vspace{5mm}
\large \textbf{Prof.Kannan M. Moudgalya} \vspace{1mm}\\ Chemical Engineering Department  \vspace{1mm} \\IIT Bombay
}
\vspace{1cm}

\maketitle


\newpage
\title{\textbf{\textbf{\LARGE 
\begin{flushleft}
\textbf{Acknowledgment}
\end{flushleft}
}}}
\large We are extremely thankful to Prof . Kannan Moudgalya for guiding and motivating us throughout the FOSSEE fellowship programme.We would also like to thank our mentors Mrs.Gloria Nandihal and Mr.Athul George for their immense support and advice.Moreover, we are grateful to our fellow friends Mudit Joshi and Ashutosh Gangwar for assisting us with their programming knowledge.Lastly,we extend our warm gratiutude to the managers and staff of FOSSEE for their co-operation and assistance.

\tableofcontents

\chapter{\textbf{Introduction}}
\section{Introduction}
KiCad is an open-source software tool for the creation of electronic schematic diagrams and PCB artwork.KiCad can be considered mature enough to be used for the successful development and maintenance of complex electronic boards.KiCad does not present any board-size limitation and it can easily handle up to 32 copper layers, up to 14 technical layers and up to 4 auxiliary layers. KiCad can create all the files necessary for building printed boards, Gerber files for photo-plotters, drilling files, component location files and a lot more.Being open source (GPL licensed), KiCad represents the ideal tool for projects oriented towards the creation of electronic hardware with an open-source flavour.

Stable builds
Stable releases of KiCad can be found in most distribution’s package managers as kicad and kicad-doc. If your distribution does not provide latest stable version, please follow the instruction for unstable builds and select and install the latest stable version.

Unstable (nightly development) builds
Unstable builds are built from the most recent source code. They can
sometimes have bugs that cause file corruption, generate bad gerbers, etc, but are generally stable and have the latest  features.

\chapter{\textbf{Steps to follow for better understanding of kicad nightly builds for electronic simulation}}
\section{Steps to follow for better understanding of kicad nightly builds for electronic simulation}
\begin{itemize}
\item Download Kicad Documentation from \url{http://docs.kicad-pcb.org/stable/en/getting_started_in_kicad.pdf}
\item Go through the working of Kicad.
\item Download \url{https://github.com/KiCad/kicad-source-mirror}
\item Go to demos->simulation where you will find three example circuits
\item Figure out the working of simulations and adding library files to components as shown in the examples.
\item  Post questions on the Kicad forum if you are facing any difficulties.
\end{itemize}
\chapter{\textbf{ANALOG CIRCUITS}}
\section{ANALOG CIRCUITS}
\subsection{Second Order low pass filter}
Problem Statement:
Plot the input and output waveform of second order low pass filter and verify the same using AC Analysis

Solution:
\begin{itemize}
\item Create a schematic in Kicad as shown in the diagram below
\item Download \url{https://github.com/KiCad/kicad-source-mirror}
\item To add generic op amp model click on  create,delete and edit symbols in the top toolbox as shown in diagram.
\begin{flushleft}
\includegraphics[width = \textwidth]{1.png}
\end{flushleft}
\newpage
\item  Next click on Add an existing library in the next window
\begin{flushleft}
\includegraphics[width = 0.6\textwidth]{2.png}
\end{flushleft}
\item Go to kicad-source mirror master   demos  $\rightarrow$simulation $\rightarrow$sallen$\_$key$\rightarrow$sallen$\_$key$\_$schlib.lib
\item Finally generic op amp is added to your components library ,add it to your schematic and right click on modeledit properties$\rightarrow$edit$\rightarrow$ spice model$\rightarrow$model.Give library as 8051.lib present in Kicad source mirror and Type as subcircuit.
\item Provide ac analysis in simulation and run the schematic
\end{itemize}
Circuit
\includegraphics[width = 0.9\textwidth]{3.png}
\newpage
Waveform
\includegraphics[width = 0.9\textwidth]{4.png}
\subsection{DIFFERENCE AMPLIFIER}
Problem Statement:
Plot  the input and output waveform of difference amplifier and verify the same using transient analysis

SOLUTION:
\begin{itemize}
\item Create a Kicad schematic as shown in the diagram.
\item Use a generic op amp as opposed to any other op amp as you would need to create your own netlist for circuit to work.
\item To learn how to create netlist visit \url{ http://www.ecircuitcenter.com/Circuits/opmodel1/opmodel2.htm}
\item Netlist for difference op amp
\begin{flushleft}
\includegraphics[width = 0.8\textwidth]{5.png}
\end{flushleft}
\item Give transient analysis and run the simulator for waveforms.
\end{itemize}
\newpage
CIRCUIT
\begin{flushleft}
\includegraphics[width = 0.8\textwidth]{6.png}
\end{flushleft}
WAVEFORM
\begin{flushleft}
\includegraphics[width = 0.8\textwidth]{7.png}
\end{flushleft}
\subsection{Voltage Regulator}
Problem Statement:
Plot the input and output waveform of Voltage Regulator and verify the same using transient analysis.

Solution:
\begin{itemize}
\item Create a Kicad schematic as shown in the diagram
\item To use the transistor Q$\_$NPN$\_$CBE  in your components library ,follow the steps mentioned in example 2.2.1.
\item  Go the Kicad-source mirror master$\rightarrow$demos$\rightarrow$simulation$\rightarrow$laser$\_$driver and add  Laser$\_$driver$\_$schlib.lib
\item To add .lib file for transistor ,right click on model$\rightarrow$edit spice model$\rightarrow$add fzt1049a.lib present in laser$\_$driver and type as BJT
\item For zener diode select D$\_$Zener present in devices of components library and add netlist for the same as shown in in figure below.
\begin{flushleft}
\includegraphics[width = 0.8\textwidth]{8.png}
\end{flushleft}
\item Provide dc analysis and run the simulation
\end{itemize}
CIRCUIT
\begin{flushleft}
\includegraphics[width = 0.8\textwidth]{9.png}
\end{flushleft}
WAVEFORM
\begin{flushleft}
\includegraphics[width = 0.8\textwidth]{10.png}
\end{flushleft}

\subsection{Hartley Oscillator}
Problem Statement:
Plot the output waveform of hartley oscillator and verify the same using transient analysis

Solution:
\begin{itemize}
\item Create a Kicad schematic as shown in the diagram
\item Use BC548 transistor present in components library and add NPN.lib  present in
    libs folder in analog circuits in my github \url{https://github.com/FOSSEE/eSIm-Kicad-Simulations}
\item Use transient analysis and run simulation
\end{itemize}
CIRCUIT
\begin{flushleft}
\includegraphics[width = 0.8\textwidth]{11.png}
\end{flushleft}
WAVEFORM
\begin{flushleft}
\includegraphics[width = 0.8\textwidth]{12.png}
\end{flushleft}
\subsection{Notch Filter}
Problem Statement:
To design a notch filter that resonates at 2 kHz and verify using AC analysis in Kicad.

Solution:
\begin{itemize}
\item Create a schematic in Kicad using the circuit diagram
\item Use op-amp model AD8051 in the circuit (recommended)
\item Add .cir file of the AD8051 op-amp in Edit Sources $\rightarrow$ Model $\rightarrow$ Library.
\item Do ac analysis from a frequency range of 10 Hz to 10 Megahertz
\item Verify the resonant frequency from the simulated output plot
\end{itemize}
CIRCUIT SCHEMATIC:
\begin{flushleft}
\includegraphics[width = 0.8\textwidth]{13.png}
\end{flushleft}
WAVEFORM:
\begin{flushleft}
\includegraphics[width = 0.8\textwidth]{14.png}
\end{flushleft}

\subsection{DIFFERENTIATOR:}
Problem Statement:
To realise an op-amp differentiator using Kicad.

Solution:
\begin{itemize}
\item Create a schematic in Kicad using the circuit diagram
\item Use op-amp model AD8620 in the circuit.
\item Do transient analysis upto 100 milliseconds with a time-step of 5ms
\item Verify the simulated output plot
\end{itemize}
CIRCUIT
\begin{flushleft}
\includegraphics[width = 0.8\textwidth]{15.png}
\end{flushleft}
WAVEFORM
\begin{flushleft}
\includegraphics[width = 0.8\textwidth]{16.png}
\end{flushleft}
\begin{flushleft}
\includegraphics[width = 0.8\textwidth]{17.png}
\end{flushleft}
\subsection{PRECISION CLIPPER :}
Problem Statement:
To design a precision clipper circuit using Kicad

Solution
\begin{itemize}
\item Create a schematic in Kicad using the circuit diagram
\item Use a generic op-amp model in the circuit (recommended)
\item Add .cir file ad8051.cir  in Edit Sources $\rightarrow$ Model $\rightarrow$ Library.
\item Do transient analysis upto 100 milliseconds with a time-step of 5ms
\item Note-Do not interchange pins of diode in this case
\item Verify  the simulated output plot
\end{itemize}
CIRCUIT
\begin{flushleft}
\includegraphics[width = 0.8\textwidth]{18.png}
\end{flushleft}
WAVEFORM
\begin{flushleft}
\includegraphics[width = 0.8\textwidth]{19.png}
\end{flushleft}
\newpage
\chapter{\textbf{SUBCIRCUIT BUILDER METHOD}}
\section{SUBCIRCUIT BUILDER METHOD}
\begin{itemize}
\item Subcircuits are used to create large components from simple components.
\item Since most Kicad components such as nand gates,flip flops etc need netlists for them to function,we have found that the subcircuit method is most successful for create netlists for circuits which we want to add
\item Most digital circuits in Kicad nightly builds can be built by the method of subcircuit.
\item To explain the concept of subcircuit,I will take example of  creating a nand gate using cmos logic
\item First create a nand gate using cmos logic and basic devices as shown in figure below.
\begin{flushleft}
\includegraphics[width = 0.8\textwidth]{20.png}
\end{flushleft}
\item Generate netlist and simulate.Check whether waveforms are correct.
\item Now open a new project and click on create,editor delete symbol in toolbox
\begin{flushleft}
\includegraphics[width = 0.8\textwidth]{21.png}
\end{flushleft}
\item Next click on create a new library to store all your subcircuit files
\item Select symbol  libray table to add symbol select project
\begin{flushleft}
\includegraphics[width = 0.8\textwidth]{22.png}
\end{flushleft}
\item Click on create new symbol and select the subcircuit file you have created 
\begin{flushleft}
\includegraphics[width = 0.8\textwidth]{23.png}
\end{flushleft}
\item Create the new subcircuit using toolbox provided in the right as shown in figure below
 \begin{flushleft}
\includegraphics[width = 0.8\textwidth]{24.png}
\end{flushleft}
\item Next step is to create a subckt file for cmos$\_$nand using the netlist generated copy the contents as highlighted in the figure below
 \begin{flushleft}
\includegraphics[width = 0.8\textwidth]{25.png}
\end{flushleft}
\item Next create a subcircuit  .lib file for the netlist generated using .subckt and adding pin numbers.Make sure pin numbers for subcircuit is same as that provided for the circuit.Use globel labels for providing pin numbers.Generated subcircuit would look as follows.
 \begin{flushleft}
\includegraphics[width = 0.8\textwidth]{26.png}
\end{flushleft}
\item Now you are free to use the subcircuit created by you and use it as a component for the required functionality.
\end{itemize}
CIRCUIT
 \begin{flushleft}
\includegraphics[width = 0.8\textwidth]{27.png}
\end{flushleft}
WAVEFORM
\begin{flushleft}
\includegraphics[width = 0.8\textwidth]{28.png}
\end{flushleft}
\chapter{\textbf{DIGITAL CIRCUITS}}
\section{DIGITAL CIRCUITS}
\subsection{Transient Amplifier}
 Problem Statement:
Plot the input and output waveforms of transient amplifier and verify the same using transient analysis

SOLUTION:
\begin{itemize}
\item Create a circuit as shown in the diagram.
\item Use basic components and there is no need for subcircuit in this circuit.
\item Give transient analysis and simulate the circuit
\end{itemize}
CIRCUIT:
 \begin{flushleft}
\includegraphics[width = 0.8\textwidth]{29.png}
\end{flushleft}
WAVEFORM
 \begin{flushleft}
\includegraphics[width = 0.8\textwidth]{30.png}
\end{flushleft}
\subsection{D Flip Flop}
Problem Statement:
Plot the input and output waveforms of D Flip Flop and verify the same using Transient Analysis

Solution:
\begin{itemize}
\item Create a schematic in Kicad as shown in the diagram
\item Use the subcircuit Builder method as mentioned in section 4.1.1 to build cmos$\_$nand gates and add it to your component library
\item Make sure to use correct pins in building subcircuit
\item Add library file to each and every nand gate
\item Recommended to use pwl for clock instead of pulse.
\item Provide transient analysis and run simulation
\end{itemize}

Circuit
 \begin{flushleft}
\includegraphics[width = 0.8\textwidth]{31.png}
\end{flushleft}

Waveform
 \begin{flushleft}
\includegraphics[width = 0.8\textwidth]{32.png}
\end{flushleft}

\subsection{D Latch}
Problem Statement:
Plot the input and output waveform of D Latch and verify the same using transient analysis

Solution:
\begin{itemize}
\item Create a schematic for D Latch as shown in the diagram.
\item Use the subcircuit builder method to build  cmos nand gate and add it to you components library
\item Make sure pins are correct and provide correct sources.
\item Give transient analysis and run the simulation
\end{itemize}

Circuit
 \begin{flushleft}
\includegraphics[width = 0.8\textwidth]{33.png}
\end{flushleft}

Waveform
 \begin{flushleft}
\includegraphics[width = 0.8\textwidth]{34.png}
\end{flushleft}

\subsection{ Masterslave JK Flip Flop}
Problem Statement:
Plot the input and output waveforms of mastersave JK flip flop and verify the same using transient analysis

Solution
\begin{itemize}
\item Create the schematic as shown in the diagram
\item Use subcircuit builder method to buid and add  cmos$\_$nand gate.
\item Make sure the numbering of pins is correct
\item Note that I have made a positive edge  triggered JK Flip Flop by adding not gate in the beginning
\item Provide transient analysis and run the simulation
\end{itemize}

Circuit
\begin{flushleft}
\includegraphics[width = 0.8\textwidth]{35.png}
\end{flushleft}

 Subcircuit
\begin{flushleft}
\includegraphics[width = 0.8\textwidth]{37.png}
\end{flushleft}

Waveform
\begin{flushleft}
\includegraphics[width = 0.8\textwidth]{36.png}
\end{flushleft}

\subsection{Full Adder}
Problem Statement:
Plot the input and out waveforms for Full adder and verify the same using traansient analysis

Solution:
\begin{itemize}
\item Create the schematic as shown in the diagram
\item You must first create subcircuit for and gate and xor gate using subcircuit builder method
\item Note the output might have little fluctuations,you must ignore this.
\item Provide transient analysis and run the simulation.
\end{itemize}

Circuit
\begin{flushleft}
\includegraphics[width = 0.8\textwidth]{38.png}
\end{flushleft}
Waveform
\begin{flushleft}
\includegraphics[width = 0.8\textwidth]{39.png}
\end{flushleft}

\subsection{Johnson Counter}
Problem Statement
Plot the input and output waveform of johnson counter and verify the same using transient analysis

Solution
\begin{itemize}
\item Create a schematic in Kicad using the circuit diagram
\item Use sub circuit of d-flip flop
\item Add .lib file of the flip flop in Edit Sources $\rightarrow$ Model $\rightarrow$Library.
\item Do transient analysis upto 100 milliseconds with a time-step of 5ms
\item Verify the counter value with the no. of clock pulses
\end{itemize}

Circuit
\begin{flushleft}
\includegraphics[width = 0.8\textwidth]{40.png}
\end{flushleft}

Waveform
\begin{flushleft}
\includegraphics[width = 0.8\textwidth]{41.png}
\end{flushleft}
\newpage

\chapter{\textbf{UPLOADED PROJECTS}}
\section{UPLOADED PROJECTS}
\begin{itemize}
\item All projects given in analog and digital circuits and more has been uploaded in \url{https://github.com/FOSSEE/eSIm-Kicad-Simulations}
\item Kindly go through these examples for better understanding of kicad simulations,adding and creating netlists and building subcircuits
\item Libs folder in both analog and digital circuits contains the list of netlists for each model used in the simulations
\item In digital circuits sim$\_$logic.lib contains the list of subckts made.You can add these files to your component library for using in your digital circuits.
\item The library file sim$\_$model.lib contains the list of netlists for digital circuits.
\end{itemize}
\chapter{\textbf{Problems faced and their solutions}}
\section{Problems faced and their solutions}
Problem:Circuits consisting of diodes were not giving correct waveforms.

Solution:Since diode is a basic component there is no need of a .lib file added to it.However we must interchange the position of the pins for the diode to work correctly.For Zener diode we must both interchange position and add netlist
\begin{flushleft}
\includegraphics[width = 0.8\textwidth]{42.png}
\end{flushleft}

Problem:Some circuits involving transistors,op amps,zener diodes were not  producing correct waveforms.

Solution:Add the corresponding netlist for each of these devices.You may have to build the netlist if it is not available or use subcircuit builder method.

Problem:We were not aware on how to give value for pwl sources

Solution:We have to add a delay whenever there is a transition

Problem:Digital circuits were not working properly

Solution:Use subcircuit builder method

\begin{flushleft}
\includegraphics[width = 0.8\textwidth]{43.png}
\end{flushleft}
\chapter{\textbf{Conclusion}}
\section{Conclusion}
\begin{itemize}
\item We have simulated analog and digital circuits using Kicad nightly builds and have uploaded on Github
\item We have also made subciruits and libraries for many components.
\end{itemize}
\newpage

\title{\textbf{\textbf{\LARGE 
\begin{flushleft}
\textbf{Reference}
\end{flushleft}
}}}




\begin{itemize}
\item \url{http://docs.kicad-pcb.org/stable/en/getting_started_in_kicad.pdf}
\item \url{https://github.com/KiCad/kicad-source-mirror}
\item  \url{https://github.com/FOSSEE/eSIm-Kicad-Simulations}
\item \url{ https://forum.kicad.info/}
\item \url{ http://www.ecircuitcenter.com/Basics.htm}
\item \url{http://www.ecircuitcenter.com/SPICEsummary.htm}
\end{itemize}
\end{document}



