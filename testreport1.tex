\documentclass[12pt,a4paper]{  report}
\usepackage[hmargin=3cm,vmargin=3cm]{geometry}
\usepackage{graphicx}
\usepackage{caption}
\usepackage{array}
\usepackage{listings}
\usepackage{hyperref}
\usepackage{mathtools}
\usepackage{textcomp}
\graphicspath{Images}
\begin{document}
\begin{figure}
\centering
\includegraphics[width = 0.3\textwidth]{iit.png}
\hspace{1cm}
\includegraphics[width = 0.4\textwidth]{fossee-logo.png}
\end{figure}

\title{\textbf{\textbf{Summer Fellowship Report}}\vspace{5mm} \\\small On \\\vspace{5mm} \textbf{\large Spice Simulations using Kicad nightly builds}\vspace{5mm} \\ \vspace{5mm}\small Submitted by\\  \vspace{5mm}  \large \textbf{Akshay NH  and Athul MS}\\ \vspace{5mm}
\small Under the guidance of \\ \vspace{5mm}
\large \textbf{Prof.Kannan M. Moudgalya} \vspace{1mm}\\ Chemical Engineering Department  \vspace{1mm} \\IIT Bombay
}
\vspace{1cm}

\maketitle


\newpage
\title{\textbf{\textbf{\LARGE 
\begin{flushleft}
\textbf{Acknowledgment}
\end{flushleft}
}}}
\large We are extremely thankful to Prof . Kannan Moudgalya for guiding and motivating us throughout the FOSSEE fellowship programme.We would also like to thank our mentors Mrs.Gloria Nandihal and Mr.Athul George for their immense support and advice.Moreover, we are grateful to our fellow friends Mudit Joshi and Ashutosh Gangwar for assisting us with their programming knowledge.Lastly,we extend our warm gratiutude to the managers and staff of FOSSEE for their co-operation and assistance.

\tableofcontents

\chapter{\textbf{Introduction}}
\section{Introduction}
KiCad is an open-source software tool for the creation of electronic schematic diagrams and PCB artwork.KiCad can be considered mature enough to be used for the successful development and maintenance of complex electronic boards.KiCad does not present any board-size limitation and it can easily handle up to 32 copper layers, up to 14 technical layers and up to 4 auxiliary layers. KiCad can create all the files necessary for building printed boards, Gerber files for photo-plotters, drilling files, component location files and a lot more.Being open source (GPL licensed), KiCad represents the ideal tool for projects oriented towards the creation of electronic hardware with an open-source flavour.

Stable builds
Stable releases of KiCad can be found in most distribution’s package managers as kicad and kicad-doc. If your distribution does not provide latest stable version, please follow the instruction for unstable builds and select and install the latest stable version.

Unstable (nightly development) builds
Unstable builds are built from the most recent source code. They can
sometimes have bugs that cause file corruption, generate bad gerbers, etc, but are generally stable and have the latest  features.

\chapter{\textbf{Steps to follow for better understanding of kicad nightly builds for electronic simulation}}
\section{Steps to follow for better understanding of kicad nightly builds for electronic simulation}
\begin{itemize}
\item Download Kicad Documentation from \url{http://docs.kicad-pcb.org/stable/en/getting_started_in_kicad.pdf}
\item Go through the working of Kicad.
\item Download \url{https://github.com/KiCad/kicad-source-mirror}
\item Go to demos->simulation where you will find three example circuits
\item Figure out the working of simulations and adding library files to components as shown in the examples.
\item  Post questions on the Kicad forum if you are facing any difficulties.
\end{itemize}

\newpage

\title{\textbf{\textbf{\LARGE 
\begin{flushleft}
\textbf{Reference}
\end{flushleft}
}}}




\begin{itemize}
\item \url{http://docs.kicad-pcb.org/stable/en/getting_started_in_kicad.pdf}
\item \url{https://github.com/KiCad/kicad-source-mirror}
\item  \url{https://github.com/FOSSEE/eSIm-Kicad-Simulations}
\item \url{ https://forum.kicad.info/}
\item \url{ http://www.ecircuitcenter.com/Basics.htm}
\item \url{http://www.ecircuitcenter.com/SPICEsummary.htm}
\end{itemize}
\end{document}



